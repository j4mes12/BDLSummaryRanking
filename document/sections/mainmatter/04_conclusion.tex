{\bf A compulsory chapter,  roughly 10\% of the total page-count}
\vspace{1cm} 

\noindent
The concluding chapter(s) of a dissertation are often underutilized because they're 
too often left too close to the deadline: it is important to allocate enough time and 
attention to closing off the story, the narrative, of your thesis.

Again, there is no single correct way of closing a thesis. 

One good way of doing this is to have a single chapter consisting of three parts:

\begin{enumerate}
	\item (Re)summarise the main contributions and achievements, in essence
	      summing up the content.
	\item Clearly state the current project status (e.g., ``X is working, Y 
	      is not'') and evaluate what has been achieved with respect to the 
	      initial aims and objectives (e.g., ``I completed aim X outlined 
	      previously, the evidence for this is within Chapter Y'').  There 
	      is no problem including aims which were not completed, but it is 
	      important to evaluate and/or justify why this is the case.
	\item Outline any open problems or future plans.  Rather than treat this
	      only as an exercise in what you {\em could} have done given more 
	      time, try to focus on any unexplored options or interesting outcomes
	      (e.g., ``my experiment for X gave counter-intuitive results, this 
	      could be because Y and would form an interesting area for further 
	      study'' or ``users found feature Z of my software difficult to use,
	      which is obvious in hindsight but not during at design stage; to 
	      resolve this, I could clearly apply the technique of Bloggs {\em et al.}.
\end{enumerate}

Alternatively, you might want to divide this content into two chapters: a penultimate chapter with a title such as ``Further Work" and then a final chapter ``Conclusions". Again, there is no hard and fast rule, we trust you to make the right decision. 
		
		And this, the final paragraph of this thesis template, is just a bunch of citations, added to show how to generate a BibTeX bibliography. Sources that have been randomly chosen to be cited here include:
		\cite{miller_etal_2018_clojure,webber_marwan_2015,touretzky_2013_lisp,eckmann_etal_1987,marwan_2011,vach_2015,shiller_2017,vytelingum_2006,tesfatsion_2002,rust_etal_1992}.
